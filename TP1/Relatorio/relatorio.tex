%
% Layout retirado de http://www.di.uminho.pt/~prh/curplc09.html#notas
%
\documentclass{report}
\usepackage[portuges]{babel}
\usepackage[utf8]{inputenc}
\usepackage{url}
\usepackage{enumerate}
\usepackage{fancyvrb}
 
%\usepackage{alltt}
%\usepackage{fancyvrb}
\usepackage{listings}
%LISTING - GENERAL
\lstset{
    basicstyle=\small,
    numbers=left,
    numberstyle=\tiny,
    numbersep=5pt,
    breaklines=true,
   frame=tB,
    mathescape=true,
    escapeinside={(*@}{@*)}
}
 
\usepackage{xspace}
 
\parindent=0pt
\parskip=2pt
 
\setlength{\oddsidemargin}{-1cm}
\setlength{\textwidth}{18cm}
\setlength{\headsep}{-1cm}
\setlength{\textheight}{23cm}
 
\def\darius{\textsf{Darius}\xspace}
\def\antlr{\texttt{AnTLR}\xspace}
\def\pe{\emph{Publicação Eletrónica}\xspace}
 
\def\titulo#1{\section{#1}}
\def\super#1{{\em Supervisor: #1}\\ }
\def\area#1{{\em \'{A}rea: #1}\\[0.2cm]}
\def\resumo{\underline{Resumo}:\\ }
 
 
 
\title{Sistemas de Representação de Conhecimento e Raciocínio\\ \textbf{Exercício 1}\\ Relatório de Desenvolvimento}
\author{\textbf{Grupo 18}\\
\\
\\
Cesário Miguel Pereira Perneta - A73883
\\
Luís Miguel Bravo Ferraz - A70824
\\
Tiago Miguel Fraga Santos - A74092
\\
Rui Pedro Barbosa Rodrigues - A74572}
\date{\today}
 
\begin{document}
 
\DefineVerbatimEnvironment{verbatim}{Verbatim}{xleftmargin=.5in}
 
\maketitle
 
\begin{abstract}
Neste relatório irá ser abordado a resolução e verificação do primeiro trabalho prático da Unidade Curricular de Sistemas de Representação de Conhecimento e Raciocínio.
Será possível encontrar uma detalhada explicação dos predicados elaborados, bem como a apresentação do código usado na representação dos predicados fundamentais, que visam dar resposta aos requisitos propostos neste trabalho prático.
\end{abstract}
 
\tableofcontents

\chapter{Introdução} \label{intro}

Como primeiro exercício desta unidade curricular, temos como objectivo desenvolver um sistema de representação de conhecimento e raciocínio que tenha a capacidade de caracterizar um universo de discurso na área da prestação de cuidados de saúde. Para tal usou-se o método sugerido no enunciado para a caracterização de conhecimento, onde temos \textbf{utente}, \textbf{prestador} e \textbf{cuidado}, os quais foram usados para a demonstração das várias funcionalidades. Em baixo, é possivel visualizar a forma como está definida a nossa \textbf{base de conhecimento}.

\titulo{Base de Conhecimento}
\bigbreak
\textbf{Utentes:}
\begin{verbatim}
utente: #IdUt, Nome, Idade, Morada -> {V,F}

utente(1,'Tiago',23,morada('Rua 1','Esposende')).
utente(2,'Cesario',21,morada('Rua 2','Valença')).
utente(3,'Luis',22,morada('Rua 3','Viana')).
utente(4,'Rui',21,morada('Rua 4','Braga')).
\end{verbatim}

\textbf{Prestadores:}
\begin{verbatim}
prestador: #IdPrest, Nome, Especialidade,Instituição -> {V,F}

prestador(1,'Nelon Puga','Desporto',instituicao('Clinica do Dragao', morada('Rua das Antas','Porto'))).
prestador(2,'Joaquim','Cardiologia',instituicao('Clinica do Dragao', morada('Rua das Antas','Porto'))).
\end{verbatim}

\textbf{Cuidados Médicos:}
\begin{verbatim}
cuidado: Data, #IdUt, #IdPrest, Descrição, Custo -> {V,F}

cuidado(data(10,02,2018),1,9,'Consulta Rotina',30).
cuidado(data(10,02,2018),2,9,'Consulta Rotina',30).

\end{verbatim}

\textbf{Extra - Enuncidado}
\bigbreak
Além do enunciado proposto, decidimos considerear estes dois predicados extra, de forma a oferecer mais dinamismo e, principalmente, organização na nossa base de conhecimento de forma a que a seleção de informação fosse feita de uma forma mais rica e sistemática.
\bigbreak
\textbf{Instituições:}
\begin{verbatim}
instituição: Data, #IdUt, #IdPrest, Descrição, Custo -> {V,F}

instituicao('Clinica do Dragao', morada('Rua das Antas','Porto')).
instituicao('Hospital Sao Joao', morada('Rua da Circunvalacao','Porto')).
\end{verbatim}

\textbf{Moradas:}
\begin{verbatim}
morada: Rua,Localidade -> {V,F}

\end{verbatim}

\textbf{Datas:}
\begin{verbatim}
data: Dia, Mes, Ano -> {V,F}

data(D, M, A) :- pertence(M, [1,3,5,7,8,10,12]), D >= 1, D =< 31.
data(D, M, A) :- pertence(M, [4,6,9,11]), D >= 1, D =< 30.
data(D, 2, A) :- A mod 4 =\= 0, D >= 1, D =< 28.
data(D, 2, A) :- A mod 4 =:= 0, D >= 1, D =< 29.
\end{verbatim}


\chapter{Desenvolvimento} \label{intro}
 

Neste capitulo será explicado os predicados utilizados na resolução dos requisitos propostos. Apresentar-se-à também o código usado na representação dos predicados fundamentais, e ainda uma breve explicação acerca dos predicados auxiliares utilizados.
 













\titulo{Registar utentes, prestadores e cuidados médicos.}

Registo na base de conhecimento os predicados utentes, prestadores e cuidados médicos. 

\subsection{Predicado Principal}
 
\begin{verbatim}
resgistar T -> {V,F}
\end{verbatim}
 
O registo de utentes,prestadores e cuidados prestados é efetuado a partir do predicado registar. 
 

\begin{verbatim}
registar( Termo ) :-
    solucoes( Invariante,+Termo::Invariante,Lista), 
                insercao(Termo), 
                    teste(Lista).
\end{verbatim}
 
Este predicado tem em conta os invariantes criados, de modo a evitar a inserção de conhecimento repetido, e de modo a manter a consistência da nossa base de conhecimento. Os invariantes utilizados para o efeito são os seguintes:    

\subsection{Invariantes}

\begin{verbatim}
+utente(ID,N,I,M) :: 
        (solucoes( (ID), 
            (utente(ID,Ns,Is,Ms)),S), 
                comprimento(S,N), N==1).
\end{verbatim}
Este invariante está definido de forma a não permitir informação repetida de utentes na nossa base de conhecimento. Pode estar repetido através do ID, ou seja, cada utente tem um ID único.
\bigbreak

\begin{verbatim}
+prestador(ID,N,E,I) :: 
        (solucoes( (ID), 
            (prestador(ID,Ns,Es,Is)), S),
                comprimento(S,N),N==1).
\end{verbatim}
Este invariante está definido de forma a não permitir informação repetida de prestadores na nossa base de conhecimento. Pode estar repetido através do ID, ou seja, cada prestador tem um ID único.
\bigbreak

\begin{verbatim}
+cuidado(Dt,IDU,IDP,D,C) :: 
        (solucoes( (Dt,IDU,IDP),
            (cuidado(Dt,IDU,IDP,Ds,Cs)),S),
                comprimento(S,N),N==1).
\end{verbatim}
Este invariante está definido de forma a não permitir informação repetida de cuidados de saude na nossa base de conhecimento. Pode estar repetido através da data, do IDU e do IDP em simultâneo, ou seja, cada cuidado de saude tem um trio data, IDU e IDP único.
\bigbreak

\begin{verbatim}
+cuidado(Dt,IDU,IDP,D,C) :: 
        (solucoes( (IDU),
            (utente(IDU,Ns,Is,Ms)),S),
                comprimento(S,N),N==0).

+cuidado(Dt,IDU,IDP,D,C) :: 
        (solucoes( (IDP),
            (prestador(IDP,Ns,Es,Is)),S),
                comprimento(S,N),N==0).
\end{verbatim}
Estes dois invariantes protegem a nossa base de conhecimento, de maneira a não ser inserido conhecimento por falta de conhecimento, ou seja, nao é possivel adicionar um cuidado de saude com um determinado IDU ou IDP caso estes nao constem no predicado utente ou prestador, respetivamente.
\bigbreak

\begin{verbatim}
+instituicao(N,M) :: 
    (solucoes( (N), 
        instituicao(N,M) ,S), 
            N==1).
\end{verbatim}
Este invariante está definido de forma a não permitir informação repetida de instituições na nossa base de conhecimento. Pode estar repetido através do nome, ou seja, cada instituição tem um nome único.
\bigbreak

\subsection{Predicados Auxiliares}

Nesta secção apresentamos os predicados auxiliares usados no predicado registar.


\begin{verbatim}
solucoes T,Q,S -> {V,F}
\end{verbatim}

Predicado que guarda numa lista as soluções da questão efetuada,com o formato especificado.

\begin{verbatim}
insercao T -> {V,F}

insercao(T) :- assert(T).
insercao(T) :- retract(T),!,fail.
\end{verbatim}

Este predicado trata de inserir na base de conhecimento em caso de sucesso, senão remove o conhecimento inserido em caso de insucesso.

\begin{verbatim}
teste L -> {V,F}

teste([]).
teste([I|L]) :- I, teste(L).
\end{verbatim}

Testa se existe algum invariante que não seja satisfeito, se falhar em algum dos invariantes , a inserção falha e assim não ocorre evolução da nossa base de conhecimento. Se todos os invariantes forem satisfeitos, a nossa base de conhecimento é evoluída com sucesso.

\begin{verbatim}
comprimento Lista, R -> {V,F}

comprimento([],0).
comprimento([X|L],R) :- comprimento(L,N) , R is 1+N.
\end{verbatim}
Este predicado calcula o numero de elementos de uma lista.
\bigbreak





























\titulo{Remover utentes, prestadores e cuidados médicos.}

Remover na base de conhecimento os predicados utentes, prestadores e cuidados médicos. 

\subsection{Predicado Principal}
 
\begin{verbatim}
remover T -> {V,F}
\end{verbatim}
 
A remoção de utentes,prestadores e cuidados prestados é efetuado a partir do predicado remover. 
 

\begin{verbatim}
remover( Termo ) :-
    solucoes( Invariante,+Termo::Invariante,Lista), 
                remocao(Termo), 
                    teste(Lista).
\end{verbatim}

Este predicado tem em conta os invariantes criados, de modo a evitar a remoção de conhecimento critico, e de modo a manter a consistência da nossa base de conhecimento. Os invariantes utilizados para o efeito são os seguintes:    

\subsection{Invariantes}

\begin{verbatim}
-utente(ID,N,I,M) :: 
    (solucoes( (ID), 
        (cuidado(Dts,ID,IDPs,Ds,Cs)) , S),
            comprimento(S,N),N>0).
\end{verbatim}
Este invariante está definido de forma a não permitir remoção de utentes, quando estes pertencem ao predicado do conhecimento de cuidados medicos.
\bigbreak

\begin{verbatim}
-prestador(ID,N,E,I) :: 
    (solucoes( (ID), 
        (cuidado(Dts,IDUs,ID,Ds,Cs)), S),
            comprimento(S,N),N==0).
\end{verbatim}
Este invariante está definido de forma a não permitir remoção de prestadores, quando estes pertencem ao predicado do conhecimento de cuidados medicos.
\bigbreak

\begin{verbatim}
-cuidado(Dt,IDU,IDP,D,C) :: 
    (solucoes( (Dt,IDU,IDP,D,C), 
        (cuidado(Dt,IDU,IDP,D,C)) ,S),
            comprimento(S,N), N==0).
\end{verbatim}
Este invariante está definido de forma a não permitir remoção de cuidados, quando estes nao existem na base de conhecimento.
\bigbreak

\begin{verbatim}
-instituicao(N,M) :: 
    (solucoes( (N,M), 
        prestador(_,_,_,(N,M)) , S), 
            comprimento(S,N),N==0).
\end{verbatim} 
Este invariante está definido de forma a não permitir remoção de instituições, quando estes pertecem ao predicado do conhecimento de prestadores.
\bigbreak

 
\subsection{Predicados Auxiliares}

Nesta secção apresentamos os novos predicados auxiliares usados no predicado remover.

\begin{verbatim}
remocao T -> {V,F}

remocao(T) :- retract(T).
remocao(T) :- assert(T),!,fail.
\end{verbatim}



























\titulo{Identificar utentes por critérios de seleção}

Na terceiro requisito do exercício é pedido que seja possível identificar os utentes pelos seus critérios de seleção, ou seja, pelo seu \textbf{ID}, \textbf{nome}, a sua \textbf{idade} ou \textbf{morada}. Através do predicado \textit{soluções} a resposta bastante simples de concretizar.

\begin{verbatim}
ponto_tres : Critério,R -> {V,F}
\end{verbatim}

A conclusão deste predicado é instruida com critério pelo qual vai ser feita a seleção dos utentes, e uma lista aonde se irá colocar o resultado dessa mesma seleção.

\begin{verbatim}
ponto_tres(ids,R) :- 
    solucoes((ID),utente(ID,Ns,Is,Ms),R).

ponto_tres(nomes,R) :- 
    solucoes((N),utente(IDs,N,Is,Ms),R).

ponto_tres(idades,R) :- 
    solucoes((I),utente(IDs,Ns,I,Ms),R).

ponto_tres(moradas,R) :- 
    solucoes((M),utente(IDs,Ns,Is,M),R).
\end{verbatim}

Irá colocar todos os \textbf{ID's}, \textbf{nomes}, \textbf{idades} ou \textbf{moradas} presentes nos \textit{utentes} em R.
























\titulo{Identificar as instituições prestadoras de cuidados de saúde}

Identificar as instituições prestadoras de serviços de saude que ja tenham ou nao prestado cuidados de saude, permitindo fornecer ao utilizador uma noção das instituições a que pode recorrer.


\begin{verbatim}
ponto_quatro : Critério,R -> {V,F}
\end{verbatim}

Neste predicado os critérios instruidos na conclusão especificam se queremos todas as instituições presentes na base de conhecimento, ou apenas aquelas que ja forneceram cuidados de saude.

\begin{verbatim}
ponto_quatro(todas,R) :- 
    solucoes( (N) , 
        instituicao(N,M), R).

ponto_quatro(cuidados,R) :- 
    solucoes( (N), 
        (cuidado(_,_,IDP,_,_),prestador(IDP,_,_,instituicao(N,_)))S),
            unicos(S,R).
\end{verbatim}

No primeiro caso, é uma pesquisa simples com o auxilio do predicado \textit{soluções} colocando em \textbf{R}, a lista com o \textbf{nome} de todas as instituições presentes na nossa base de conhecimento. Achamos suficiente caracterizar as instituições apenas com o nome visto que ja fornece ao utilizador um nivel pormenorizado de informação.
No segundo caso, a pesquisa do predicado \textit{soluções} é mais complexa. Neste caso é feita uma compatibilidade interna de todos os cuidados de saude e todos os prestadores que contenham o mesmo \textbf{IDP}. A todos os casos que obedeçam a este critério, o nome da institução dos prestadores é guardado numa lista auxiliar, sendo posteriormente essa lista filtrada através do predicado \textit{unicos} (predicado que será definido mais à frente), colocando assim o resultado final sem repetições no argumento \textbf{R}. 












\titulo{Identificar cuidados de saúde prestados por instituição/cidade/datas}

Este ponto do enunciado do enunciado têm como função filtrar os cuidados de saúde dado uma instituição, uma cidade ou uma data.

\begin{verbatim}
ponto_cinco : Critério,X,R -> {V,F}
\end{verbatim}

No predicado acima indicado têm como critérios possíveis uma \textit{instituicao}, uma \textit{cidade} ou uma \textit{data}, sendo X, respetivamente, o \textbf{nome}, a \textbf{localidade}, ou a \textbf{data}.

\begin{verbatim}
ponto_cinco(inst,X,R) :- 
    solucoes( (Dt,IDU,IDP,D,C) , 
        (prestador(IDP,_,_,instituicao(X,_)) , cuidado(Dt,IDU,IDP,D,C) ), R).



ponto_cinco(cidade,X,R) :- 
    solucoes( (Dt,IDU,IDP,D,C) , 
        (prestador(IDP,_,_,instituicao(_,morada(_,X))) , cuidado(Dt,IDU,IDP,D,C) ), R).

% *** outra_versao -> que nao faz tanto sentido ***

% ponto_cinco(cidade,X,R) :- 
    solucoes( (Dt,IDU,IDP,D,C) , 
        (utente(IDU,_,_,morada(_,X)) , cuidado(Dt,IDU,IDP,D,C) ), R).



ponto_cinco(data,X,R) :- 
    solucoes( (X,IDU,IDP,D,C) , 
        cuidado(X,IDU,IDP,D,C) , R).
\end{verbatim}

Como é possível observar, o seu funcionamento é semelhante ao predicado anterior. No primeiro argumento do predicado soluções colocamos o que prentiamos que fosse o nosso resultado e no segundo argumento fizemos as compatibilidades necessárias para obter a lista com os resultados. Neste caso, decidimos que os argumentos minimos para oferecer ao utilizador informação suficiente sobre os cuidados de saude seriam todos os argumentos que definem o predicado \textit{cuidado}. A lista com o resultado obtido é colocada no argumento \textbf{R} da conclusão.



















\titulo{Identificar os utentes de um prestador/especialidade/instituição}

Este ponto tem como função executar a filtragem de todos os utentes que tenham efetuado cuidados de saude com um prestador, numa especialidade ou numa instituição.

\begin{verbatim}
ponto_seis : Critério,X,R -> {V,F}
\end{verbatim}

A conclusão deste predicado, tem como critérios possiveis, o \textit{prest}, a \textit{esp} e a \textit{inst}, que correspondem, respetivamente, ao \textbf{ID} do prestador, ao \textbf{nome} da especialidade e ao \textbf{nome} da instituição. Decidimos instruir a conclusão com o \textit{ID} do prestador e o nome da instituição, pois são os identificadores unicos dos prestadores e das instituições na nossa base de conhecimento, ou seja, nao existe dois prestadores com o mesmo \textit{id}, nem duas instituições com o mesmo nome.

\begin{verbatim}
ponto_seis(prest,X,R) :- 
    solucoes( (IDU,N), 
        ( cuidado(_,IDU,X,_,_),utente(IDU,N,_,_)), S), 
            unicos(S,R).



ponto_seis(esp,X,R) :- 
    solucoes( (IDU,N), 
        ( prestador(IDP,_,X,_), cuidado(_,IDU,IDP,_,_), utente(IDU,N,_,_)), S), 
            unicos(S,R).



ponto_seis(inst,X,R) :- 
    solucoes( (IDU,N), 
        ( cuidado(_,IDU,_,_,instituicao(X,_)), utente(IDU,N,_,_)), S),
            unicos(S,R).
\end{verbatim}

Na resposta a este ponto, que é colocada no argumento \textbf{R}, decidimos que cada utente ia ser caracterizado pelo seu \textbf{nome} e \textbf{ID}. Apesar do ID ser o identificador unico dos utentes na nossa base de conhecimento, achamos que fornecer apenas essa informação ao utilizador podia ser curto, portanto acrescentamos o nome de cada utente ao seu identificador.














\titulo{Identificar cuidados de saúde realizados por utente/instituição/prestador}

Este ponto, tem como objetivo listar todos os cuidados de saúde que tenham sido realizados por um utente, uma instituição ou por um prestador.

\begin{verbatim}
ponto_sete : Criterio,X,R -> {V,F}
\end{verbatim}

A conclusão deste predicado, tem como critérios possiveis o \textit{utente}, a \textit{inst} ou o \textit{prest}, que correspondem, respetivamente, ao \textbf{ID} do utente, \textbf{nome} da instituição ou ao \textbf{ID} do prestador. Como foi dito anteriormente, tanto o id do utente, o id do prestador, e o nome da instituição, são respetivamente, os identificadores unicos dos seus predicados, não existindo estes identificadores repetidos na nossa base de conhecimento, é portanto, os argumentos necessários para instruir a conclusão deste predicado.

\begin{verbatim}
ponto_sete(utente,X,R) :- 
    solucoes( (Dt,X,IDP,D,C), 
        cuidado(Dt,X,IDP,D,C), R).



ponto_sete(inst,X,R) :- 
    solucoes( (Dt,IDU,IDP,D,C), 
        ( prestador(IDP,_,_,instituicao(X,_)), cuidado(Dt,IDU,IDP,D,C)),R).



ponto_sete(prest,X,R) :- 
    solucoes( (Dt,IDU,X,D,C), 
        cuidado(Dt,IDU,X,D,C), R).
\end{verbatim}

Uma vez mais, o funcionamento do predicado soluções é igual aos descritos em cima, e o resultado colocado em \textbf{R} é a lista de todos os cuidados pretendidos, sendo que cada cuidado é definido por todos os seus argumentos.


















\titulo{Determinar todas as instituições/prestadores a que um utente já recorreu}

Este ponto tem como objetivo filtrar todas as instituições ou prestadores que ja tenham prestado cuidados de saude a um utente.


\begin{verbatim}
custoTotal : Critério,Utente,R -> {V,F}
\end{verbatim}

A conclusão deste predicado tem como critérios possiveis o \textit{inst}, caso o utilizador pretenda listar as instituições ou o \textit{prest}, caso pretenda listar os prestadores. O argumento \textbf{U} é o identificador único de um utente da nossa base de conhecimento, ou seja o seu \textbf{IDU}. 


\begin{verbatim}
ponto_oito(inst,U,R) :- 
    solucoes( (N) , 
        ( cuidado(_,U,IDP,_,_), prestador(IDP,_,_,instituicao(N,_))) ,S) , 
            unicos(S,R).


ponto_oito(prest,U,R) :- 
    solucoes( (ID,N) , 
        ( cuidado(_,U,IDP,_,_), prestador(IDP,N,_,_) ,S)) ,
             unicos(S,R).
\end{verbatim}

No argumento \textbf{R}, é colocada a lista pretendida sendo que cada instituição é definida pelo seu \textbf{nome}, e os prestadores são definidos pelo seu \textbf{ID} e pelo seu \textbf{nome}.





















\titulo{Calcular o custo total dos cuidados de saúde por utente/especialidade/prestador/datas}

Neste ponto, é pretendido que seja feito a soma total dos custos de todos os cuidados de saude efetuados por um utente, uma especialidade, um prestador, ou numa determinada data.

\begin{verbatim}
ponto_nove : Critério,Valor,R -> {V,F}
\end{verbatim}

A conslusão deste predicado tem como critérios o \textit{utente}, a \textit{esp}, o \textit{prest} e a \textit{data}. O argumento \textbf{X} é instruido com os respetivos identificadores unicos de cada predicado.


\begin{verbatim}
ponto_nove(utente,X,R) :- 
    solucoes( (C), 
        cuidado(Dt,X,IDP,D,C), S), 
            somaTotal(S,R).



ponto_nove(esp,X,R) :- 
    solucoes( (C), 
        (prestador(IDP,_,X,_), cuidado(_,_,IDP,_,C)) , S), 
            somaTotal(S,R).



ponto_nove(prest,X,R) :- 
    solucoes( (C), 
        cuidado(Dt,IDU,X,D,C), S), 
            somaTotal(S,R). 



ponto_nove(data,X,R) :- 
    solucoes( (C), 
        cuidado(X,IDU,IDP,D,C), S), 
            somaTotal(S,R).


\end{verbatim}

Após ser executada a filtragem através do predicado soluções a lista \textbf{S}, que no momento é uma lista de valores de custos, é fornecida ao prediado \textbf{somaTotal}, que calcula a soma de todos os valores e coloca esse resultado no argumento \textbf{R}.






























\titulo{Predicados Gerais}

\subsection{Predicado: nao}

Este predicado serve para negar o valor de verdade de uma questão.

\begin{verbatim}
nao: Questão -> {V,F}
\end{verbatim}

\begin{verbatim}
nao(Q) :- Q, !, fail.
nao(Q).
\end{verbatim}






\subsection{Predicado: pertence}

Este predicado foi definido para saber se um dado valor faz parte de uma lista.

\begin{verbatim}
pertence: Valor, Lista -> {V,F}
\end{verbatim}

\begin{verbatim}
pertence(H,[H|T]).
pertence(X,[H|T]) :-
    X \= H,
    pertence(X,T).

\end{verbatim}





\subsection{Predicado: unicos}

Este predicado foi definido com o obetivo de filtrar valores repetidos numa lista. É utilizado em quase todos os pontos do enunciado.

\begin{verbatim}
unicos: Lista, R -> {V,F}
\end{verbatim}

\begin{verbatim}
unicos([],[]).
unicos([H|T], R) :-
    pertence(H,T),
    unicos(T,R).
unicos([H|T], [H|R]) :-
    nao(pertence(H,T)),
    unicos(T,R).
\end{verbatim}






\subsection{Predicado: somaTotal}

Este predicado foi definido com o objetivo de calcular a soma de todos os valores de uma lista. É utilizado no predicado do ponto nove.

\begin{verbatim}
somaTotal: Lista,Resultado -> {V,F}
\end{verbatim}

\begin{verbatim}
somaTotal( [],0 ).
somaTotal( [X],X ).
somaTotal( [X|L],R ) :- somaTotal( L,R1 ), R is X+R1.
\end{verbatim}












\chapter{Conclusão} \label{concl}
 
Este primeiro exercício ajudou bastante a ambientarmo-nos à linguagem de programacão lógica PROLOG, permitindo-nos caracterizar um universo de forma simples e estruturada.\par
Os principais obstáculos que encontramos no desenvonvimento do projeto, foi na forma de como íriamos estruturar o nosso pensamento usando programacão em lógica, sendo que após ultrapassarmos este problema, a resolução do exercício foi fácil.\par
Em suma, cumprimos todos os requisitos pedidos no enunciado, conseguido assim um trabalho sólido e competente.
 
\bibliographystyle{alpha}
\bibliography{relprojLayout}
 
 
 
\end{document}