\documentclass{report}
\usepackage[portuges]{babel}
\usepackage[utf8]{inputenc}
\usepackage{url}
\usepackage{enumerate}
\usepackage{fancyvrb}
\usepackage{titlesec}
 
%\usepackage{alltt}
%\usepackage{fancyvrb}
\usepackage{listings}
%LISTING - GENERAL
\lstset{
    basicstyle=\small,
    numbers=left,
    numberstyle=\tiny,
    numbersep=5pt,
    breaklines=true,
   frame=tB,
    mathescape=true,
    escapeinside={(*@}{@*)}
}
 
\usepackage{xspace}
 
\parindent=0pt
\parskip=2pt
 
\setlength{\oddsidemargin}{-1cm}
\setlength{\textwidth}{18cm}
\setlength{\headsep}{-1cm}
\setlength{\textheight}{23cm}
 
\def\darius{\textsf{Darius}\xspace}
\def\antlr{\texttt{AnTLR}\xspace}
\def\pe{\emph{Publicação Eletrónica}\xspace}
 
\def\titulo#1{\section{#1}}
\def\super#1{{\em Supervisor: #1}\\ }
\def\area#1{{\em \'{A}rea: #1}\\[0.2cm]}
\def\resumo{\underline{Resumo}:\\ }
 
 
 
\title{Sistemas de Representação de Conhecimento e Raciocínio\\ \textbf{Exercício 3}\\ Relatório de Desenvolvimento}
\author{\textbf{Grupo 18}\\
\\
\\
Cesário Miguel Pereira Perneta - A73883
\\
Luís Miguel Bravo Ferraz - A70824
\\
Rui Pedro Barbosa Rodrigues - A74572
\\
Tiago Miguel Fraga Santos - A74092}
\date{\today}
 
\begin{document}
 
\DefineVerbatimEnvironment{verbatim}{Verbatim}{xleftmargin=.5in}
 
\maketitle
 
\begin{abstract}
Neste relatório será abordada a resolução e verificação do terceiro trabalho prático da unidade curricular de Sistemas de Representação de Conhecimento e Raciocínio.
Será possível encontrar uma detalhada explicação dos predicados elaborados, bem como a apresentaçãao dos scripts utilizados, que visam dar resposta aos requisitos propostos neste trabalho prático.
\end{abstract}
 
\tableofcontents

\chapter{Introdução} \label{intro}

Com a realização deste trabalho o objetivo da equipa docente era a motivação dos alunos para a utilização de sistemas não simbólicos na representação de conhecimento e no desenvolvimento de mecanismos de raciocínio, nomeadamente, Redes Neuronais Artificiais (RNAs) para a resolução de problemas.
Neste terceiro e último exercício pretende-se que seja realizado um estudo que envolva a identificação da qualidade de vinhos (branco e tinto) através do estudo de vários atributos pelos quais, estes são constituidos. 
No próximo capitulo vamos apresentar os \textit{datasets} e fazer uma análise pormenorizada dos mesmos e do problema em questão.


\chapter{Análise do Problema}

De forma a iniciar o trabalho prático os professores disponibilizaram dados que descrevem a qualidade de vinhos através da análise de onze factores. De forma a perceber se uma dada entrada de informação de um vinho tem um selo de qualidade correto face aos seus atributos é preciso fazer uma análise correta dos dados fornecidos, pois são estes que vamos utilizar para efetuar a aprendizagem da rede e respetivos testes. Os ficheiros sao compostos por 11 atributos sendo eles:

\bigbreak
\begin{itemize}
\item{\textbf{fixed acidity} - acidez fixa;} 
\item{\textbf{volatile acidity} - acidez volátil;}
\item{\textbf{citric acid} - acidez citrica;}
\item{\textbf{residual sugar} - açucar residual;}
\item{\textbf{chlorides} - cloretos;}
\item{\textbf{free sulfur dioxide} - dióxido livre de enxofre;}
\item{\textbf{total sulfur dioxide} - dióxido total de enxofre;}
\item{\textbf{density} - densidade;}
\item{\textbf{pH};}
\item{\textbf{sulphates} - sulfatos;}
\item{\textbf{alchohol} - alcool;} 
\end{itemize}
\bigbreak

Os dados dos vinhos foram distribuidos por dois ficheiros, um referente ao vinho tinto (\textit{winequality-red.csv}), e outro ao vinho branco (\textit{winequality-white.csv}).
De forma a procedermos a uma correta análise dos dados e efetuar trabalho sólido e competente, decidimos dividir o trabalho por três etapas, sendo elas:

\bigbreak
\begin{itemize}
\item{Análise dos Dados para a representação do conhecimento do problema;}
\item{Tratamento dos Dados para alimentar as RNA's}
\item{Topologias, aprendizagem e testes das RNA's}
\end{itemize}
\bigbreak

No primeiro ponto, pretendemos fazer uma análise aos dados fornecidos de forma a estabelcer os principais atributos para definir a qualidade dos vinhos, e com o auxilio da ferramenta \textbf{WEKA}, iremos verificar as estatiscas de cada atributo para saber de que forma influenciam a decisão final. Após a analise decidiremos se existe algum atributo a alterar ou se podemos passar para a fase seguinte com os dados inalteráveis.


No segundo ponto, iremos organizar os dados de maneiras diferentes para fornecer a diferentes redes neuronais, com o intuito que no fim possamos avaliar os resultados consoante diferentes neurónios de entrada para diferentes redes. Após conversa com o professor, decidimos que iríamos avançar com três opções. Na primeira opção pretendemos criar uma rede neuronal com 12 neurónios de entrada. A segunda opção irá ser composta por duas redes neuronais com 11 neuronios cada uma, enquanto que a terceira opção é composta por uma rede neuronal com 22 neuronios de entrada. No capitulo correto, iremos descrever pormenorizadamente cada uma destas opções.

Por fim, no terceiro ponto descrito em cima, após ter definido todos os neuronios de entrada para cada uma das redes neuronais, iremos criar cada uma das redes, designando para cada uma a sua topologia, as suas formas de aprendizagem bem como os respetivos testes com o intuito de exercitar as redes e tirar conclusões sobre os resultados obtidos.


\chapter{Análise dos Dados para a representação do conhecimento do problema}

\chapter{Tratamento dos Dados para alimentar as RNA's}

\chapter{Topologias, aprendizagem e testes das RNA's}






\chapter{Conclusão} \label{concl}

 
\bibliographystyle{alpha}
\bibliography{relprojLayout}
 
 
 
\end{document}